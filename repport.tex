\documentclass[a4paper,11pt]{article}
\usepackage[utf8]{inputenc}
\usepackage[T1]{fontenc}
\usepackage[french]{babel}
\usepackage{amsmath}
\usepackage{graphicx}
\usepackage{booktabs}
\usepackage{geometry}
\geometry{a4paper, margin=1.5cm}

\title{\textbf{Rapport de Projet : Coloration de Graphes Par Tons}}
\author{Moucer Bahdja, Abib Alicia, Guenoun Dalil}
\date{\today}

\begin{document}

\maketitle

\section{Introduction}
\subsection{Contexte}
La coloration de graphes est un problème classique en informatique et en mathématiques discrètes. 
Il consiste à contribuer une couleur à chacun de ses sommets de manière que deux sommets reliés par une arête soient de couleur différentes. On cherche souvent à utiliser le nombre minimal de couleurs, appelé \textbf{le nombre chromatique}.
Dans ce projet, nous nous intéressons à une variante particulière où la coloration doit satisfaire des conditions spécifiques liées aux distances entre les sommets.

\subsection{Problématique}
Le problème consiste à colorier un graphe non orienté de telle sorte que pour toute paire de sommets $(x,y)$, 
le nombre de couleurs communes à leurs voisinages respectifs soit inférieur ou égal à la distance entre $x$ et $y$. 
Formellement, $|\phi(x) \cap \phi(y)| \leq d(x,y)$ où $\phi(x)$ représente l'ensemble des couleurs du voisinage de $x$.

\section{Travail réalisé}

\subsection{Structure de données}
Nous avons implémenté une structure de graphe utilisant:
\begin{itemize}
\item Une matrice d'adjacence \texttt{adj} pour représenter les arêtes
\item Une matrice de distances \texttt{dist} stockant les distances entre toutes les paires de sommets
\item Un tableau de listes chaînées \texttt{coul} pour représenter les couleurs attribuées à chaque sommet
\end{itemize}

\subsection{Choix des Listes Chaînées}

Dans notre implémentation, nous avons principalement utilisé des \textbf{listes chaînées} pour représenter les ensembles de couleurs attribuées à chaque sommet.
Ce choix à plusieurs avantage pour notre problème : 

\begin{itemize}
\item \textbf{Flexibilité} : Les listes chaînées permettent une allocation dynamique de la mémoire, en fonction du nombre variable de couleurs que peut posséder un sommet.

\item \textbf{Efficacité pour les opérations fréquentes} : Les principales opérations que nous effectuons sont :
  \begin{itemize}
  \item L'ajout de nouvelles couleurs (\texttt{addfront}, \texttt{addend})
  \item La recherche d'une couleur (\texttt{lookup})
  \item Le calcul d'intersections entre ensembles de couleurs (\texttt{cardinter})
  \end{itemize}
  
\item \textbf{Implémentation légère} : Notre structure de liste chaînée est minimaliste :
\begin{verbatim}
typedef struct List List;

struct List {
    int val;    // Valeur de la couleur
    List *next; // Pointeur vers l'élément suivant
};
\end{verbatim}
\end{itemize}

\subsection{Opérations Implémentées}

Nous avons développé les fonctions de base pour manipuler ces listes :

\begin{table}[h]
\centering
\caption{Fonctions principales des listes chaînées}
\begin{tabular}{lp{8cm}}
\toprule
Fonction & Description \\
\midrule
\texttt{newitem} & Crée un nouvel élément avec une valeur donnée \\
\texttt{addfront} & Ajoute un élément en tête de liste \\
\texttt{addend} & Ajoute un élément en fin de liste \\
\texttt{lookup} & Recherche une valeur dans la liste \\
\texttt{cardinter} & Calcule le cardinal de l'intersection de deux listes \\
\texttt{freeall} & Libère toute la mémoire occupée par une liste \\
\bottomrule
\end{tabular}
\end{table}

\subsection{Justification du Choix}

Pour notre problème de coloration, les listes chaînées sont le meilleur choix car :

\begin{itemize}
\item Les ensembles de couleurs évoluent dynamiquement pendant l'exécution des algorithmes
\item Nous avons besoin fréquemment de calculer des intersections entre les différents ensembles de couleurs
\item Pas de gaspillage de la mémoire car elle est allouée au fur et à mesure des besoins
\item L'ordre des éléments dans la liste n'a pas d'importance pour notre problème
\end{itemize}

\subsection{Algorithmes implémentés}
\subsubsection{Initialisation du graphe}
\begin{itemize}
\item Génération de graphes aléatoires (\texttt{gengraph})
\item Génération de cycles (\texttt{gengraphcycle})
\item Calcul des distances via BFS (\texttt{bfs}, \texttt{initdist})
\end{itemize}

\subsubsection{Algorithmes de coloration}
\subsection{Algorithme Glouton}
\subsubsection{Principe}
L'algorithme glouton parcourt les sommets dans un ordre arbitraire et effectue pour chacun :
\begin{enumerate}
    \item Recherche de la plus petite couleur $c$ valide
    \item Vérification des contraintes de voisinage
    \item Attribution de la couleur si possible
\end{enumerate}

\subsubsection{Implémentation}
\begin{verbatim}
int colorsom(List **coul, int x) {
    int c;
    for(c = 1; convient(coul, x, c) == 0; c++)
        ;
    coul[x] = addend(coul[x], newitem(c));
    return c;
}
\end{verbatim}

\subsubsection{Complexité}
\begin{itemize}
\item Complexité temporelle : $O(n^2 \times k)$ où $k$ est le nombre chromatique
\item Complexité spatiale : $O(n)$
\end{itemize}

\subsubsection{Avantages/Limites}
\begin{itemize}
\item [+] Simple à implémenter
\item [+] Rapide pour des petits graphes
\item [-] Ne donne pas toujours la coloration optimale
\item [-] Sensible à l'ordre de parcours des sommets
\end{itemize}

\subsection{Algorithme DSATUR}
\subsubsection{Principe}
DSATUR est une heuristique qui choisit à chaque étape le sommet avec le nombre maximal de couleurs différentes dans son voisinage (saturation).

\subsubsection{Implémentation clé}
\begin{verbatim}
int dsatur(List **coul, int b) {
    int dsat[n], deg[n];
    initdeg(deg);
    memmove(dsat, deg, n * sizeof(int));
    actuvois(dsat, coul);
    s = maxdsat(dsat, deg);
    // ...
}
\end{verbatim}

\subsubsection{Complexité}
\begin{itemize}
\item Complexité temporelle : $O(n^3)$ dans le pire cas
\item Complexité spatiale : $O(n^2)$
\end{itemize}

\subsubsection{Avantages/Limites}
\begin{itemize}
\item [+] Donne généralement de meilleurs résultats que glouton
\item [+] Plus intelligent dans l'ordre de sélection des sommets
\item [-] Plus coûteux en calcul que glouton
\item [-] Nécessite des structures auxiliaires
\end{itemize}

\subsection{Algorithme Exact (Backtracking)}
\subsubsection{Principe}
L'algorithme exact utilise le backtracking pour explorer systématiquement toutes les colorations possibles jusqu'à trouver la solution optimale.

\subsubsection{Implémentation clé}
\begin{verbatim}
void colorback(List **coul, int x, int k, int *stop) {
    if(x == n) *stop = 1;
    else {
        for(c = 1; c <= k; c++) {
            if(convient(coul, x, c)) {
                coul[x] = addfront(coul[x], newitem(c));
                colorback(coul, x+1, k, stop);
                if(*stop) return;
                coul[x] = delitem(coul[x], 0);
            }
        }
    }
}
\end{verbatim}

\subsubsection{Complexité}
\begin{itemize}
\item Complexité temporelle : $O(k^n)$ (exponentielle)
\item Complexité spatiale : $O(n)$
\end{itemize}

\subsubsection{Avantages/Limites}
\begin{itemize}
\item [+] Garantit de trouver la solution optimale
\item [+] Permet de déterminer le nombre chromatique
\item [-] Impraticable pour n > 15-20 sommets
\item [-] Complexité prohibitive
\end{itemize}

\section{Performances}
\begin{table}[h]
\centering
\caption{Temps d'exécution (en ms) pour différents algorithmes}
\begin{tabular}{lccc}
\toprule
Taille du graphe & Glouton & Backtracking & DSATUR \\
\midrule
5 sommets & 2 & 15 & 5 \\
7 sommets & 3 & 120 & 8 \\
10 sommets & 5 & - & 12 \\
\bottomrule
\end{tabular}
\end{table}

\section{Conclusion}
Ce projet nous a permis de comparer les trois algorithmes de coloration de graphe par tons.
Les trois résultats montrent bien le compromis classique en algorithmique :  
plus une solution est précise plus sont coup computationnel est élevé.
L'algorithme glouton est efficace pour les solutions rapides, DSATUR offre une bonne équilibre qualité/performance, 
tandis que la méthode exacte marlgé son optimalité, sa complexité est prohibitive.

\section{Bibliographie}
\begin{itemize}
\item Les cours de graphes sur Plubel.
\item The Practice of Programming de Brian W. Kernighan et Rob Pike .
\end{itemize}

\end{document}
\documentclass[a4paper,11pt]{article}
\usepackage[utf8]{inputenc}
\usepackage[T1]{fontenc}
\usepackage{amsmath, amssymb, graphicx,hyperref, caption}
\usepackage{geometry}
\geometry{margin=1.5cm}

\title{\textbf{Mini-rapport : Coloration par tons d'un graphe}}
\author{Moucer Bahdja, Abib Alicia, Guenoun Dalil}
\date{\today}

\begin{document}

\maketitle

\section*{Introduction}
Ce projet porte sur la \textit{coloration par tons} d'un graphe, une généralisatio, de la coloration classique
où chaque sommet reçoit plusieurs couleurs. L'objectif est de déterminer, pour un graphe donné
et un paramètre $b$, le plus ptit entier $a$ tel que graphe admette une $(a,b)$-coloration oar tons.
Cetteproblématique s'inscrit dans le cadre de la théorie des graphes et trouve des applications en télécommunications
et codage.

\section*{Travail rélaisé}
Les graphes ont été codé en langage \textbf{[C++]} sous forme de matrice d'adjacence.
Des versions modifiées des algorithmes \textbf{Glouton}, \textbf{DSATUR} et \textbf{exact} ont été adaptée à la $(a,b)$-coloration : 
chaque sommet se voit attribuer $b$ couleurs, en respectant la contrainte $|\varphi(x) \cap \varphi(y)| < d(x, y)$.

\medskip
Les principale étapes : 
\begin{itemize}
    \item Génération du graphe (aléatoire ou à partir de cas classiques comme cycles, arbres, etc.)
    \item Calcul des distances $d(x,y)$ pour tous les couples de sommets
    \item Implémentation d’un algorithme glouton pour tester différentes valeurs de $a$
    \item Optimisation en réduisant les conflits et en testant plusieurs heuristiques
\end{itemize}

\end{document}